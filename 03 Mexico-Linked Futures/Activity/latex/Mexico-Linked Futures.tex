\documentclass[11pt,a4paper]{article} % Compila con lualatex o xelatex

  % --- Layout & links ---
  \usepackage[english]{babel}
  \usepackage[top=35mm, bottom=25mm, left=30mm, right=30mm]{geometry}

  \PassOptionsToPackage{hyphens}{url} % permite cortar en guiones
  \usepackage[hidelinks]{hyperref}
  \usepackage{xurl}                   % permite corte “en cualquier punto”
  \usepackage{microtype}
  \hypersetup{breaklinks=true}
  \emergencystretch=3em               % evita overfull boxes

  \usepackage[hidelinks]{hyperref}

  % --- Fuentes (texto y matemáticas) ---
  \usepackage{amsmath}  % Para escribir fórmulas matemáticas
  \usepackage{amssymb}  % Símbolos adicionales
  \usepackage{fontspec}  % Para usar fuentes del sistema
  \usepackage{newtxmath}  % Libertinus Math para fórmulas
  \setmainfont{EB Garamond}[
    UprightFont = * Regular,
    ItalicFont = * Italic,
    BoldFont = * SemiBold,
    BoldItalicFont = * SemiBold Italic
  ]

  % --- Secciones y subsecciones ---
  \usepackage{titlesec}
  \newfontface\boldd{EB Garamond Bold}
  \newfontface\bolditalic{EB Garamond  Bold Italic}
  \newfontface\extrabold{EB Garamond ExtraBold}
  \newfontface\medium{EB Garamond Medium}

  % --- Change the monostyle font to Inconsolata ---
  \setmonofont{Inconsolata}[
    Scale=MatchLowercase,
    Contextuals=Alternate,
    Ligatures=TeX
  ]



  \usepackage{bookmark}              % mejora anchors
  \hypersetup{hypertexnames=false}   % evita destinos idénticos

  \usepackage{etoolbox,needspace}

  % Rompe página solo en \section y crea ancla propia
  \pretocmd{\section}{\clearpage\phantomsection\needspace{6\baselineskip}}{}{}

  % No rompas página en \subsection; solo asegura espacio
  \pretocmd{\subsection}{\phantomsection\needspace{4\baselineskip}}{}{}

  \setcounter{secnumdepth}{0}

  % --- Hacer los títulos de sección y subsección más grandes ---
  \titleformat{\section}
    {\boldd\fontsize{34pt}{34pt}\selectfont} % el primer {} es el formato, el segundo {} es el tamaño de línea
    {\thesection}{18em}{} % el primer {} es el formato, el segundo {} es la separación entre número y título
    \titleformat{\subsection}
      {\boldd\fontsize{18pt}{18pt}\selectfont}
      {\thesubsection}{10em}{}
  \titlespacing*{\section}{0pt}{24pt}{48pt} % el primer {} es la sangría, el segundo {} es el espacio antes, el tercero {} es el espacio después
  \titlespacing*{\subsection}{0pt}{18pt}{6pt} % el primer {} es la sangría, el segundo {} es el espacio antes, el tercero {} es el espacio después

  % --- Encabezado con nombre de la sección ---
  \usepackage{fancyhdr}
  \pagestyle{fancy}
  \fancyhf{}
  \fancyhead[L]{\small\leftmark}
  \fancyhead[R]{\small\thepage}
  \fancyhead[C]{\small\textit{\rightmark}}
  \renewcommand{\headrulewidth}{0.1pt}

  %\textsc{} para versalitas

  % --- Crear un nuevo estilo de pagina con el numero arriba a la derecha ---
  \fancypagestyle{myfancy}{
    \fancyhf{}
    \fancyhead[R]{\small\thepage}
    \renewcommand{\headrulewidth}{0pt}
  }

      % --- Cambia el pagestyle a 'plain' en cada \section ---
  \let\oldsection\section
  \renewcommand{\section}{%
    \clearpage
    \thispagestyle{myfancy}%
    \oldsection
  }

  % --- Espaciado ---
  \usepackage{setspace}

  % --- Otros paquetes útiles ---
  \usepackage{tocloft}  % Para personalizar la tabla de contenidos
  \usepackage{xcolor}   % Para colores en el código
  \usepackage{booktabs} % Para tablas bonitas

  % \usepackage{amsmath}
  % \usepackage[colorlinks=true, allcolors=blue]{hyperref}

  \usepackage{graphicx}


  % --- Bibliografía ---
  \usepackage[authoryear]{natbib}
  \makeatletter
  \renewcommand{\bibsection}{\relax} % suprime el título que inserta natbib
  \makeatother
  \bibliographystyle{apalike}   % u otro: plainnat, abbrvnat, unsrtnat, ap


  % --- Para incluir PDFs ---
  \usepackage{pdfpages}

  
  % --- Código fuente ---
  \usepackage{listings}
  \lstset{
    basicstyle=\ttfamily\small,
    backgroundcolor=\color{gray!10},
    frame=single,
    breaklines=true
  }

  % interlineado de 1.2
  \usepackage{setspace}
  \setstretch{1.2}
  \setlength{\parskip}{0.2em} % espacio entre párrafos

  \begin{document}

  % --- Portada ---
  \begin{titlepage}
  \centering
  \vspace*{4cm}
  {\extrabold\fontsize{28pt}{28pt}\selectfont Mexico-Linked Futures\par}
  \vspace{1cm}
  
  %{\medium\fontsize{12pt}{12pt}\selectfont
  {Productos Derivados: O25 LAT4012 2\par}
  {Professor:  Enríque Covarrubias Jaramillo\par}
  {Heriberto Espino Montelongo, ID: 175199\par}
  {Universidad de las Américas Puebla\par}

  %}

  \vspace{8em}
  \begin{abstract}
\noindent
Comparative analysis of six most related futures to Mexico: corn, WTI crude, silver, MXN/USD (CME FX), Mexican Funding-TIIE monthly (CME IR), and IPC. For storable commodities, pricing follows \(F=S\,e^{(r+u-y)T}\), decomposing basis into funding \(r\), operating/storage \(u\), and convenience yield \(y\) (in this work, $u=y=0$); for FX, covered interest parity \(F=S\,e^{(r_{\mathrm{USD}}-r_{\mathrm{MXN}})T}\); for rates, settlement on the compounded monthly F-TIIE index. An interpretation of the term structure and one-day pricing decomposition is provided for each asset, along with Mexico-specific implications. 

  \end{abstract}

  \vfill  
  {September 17, 2025 \par}
  \end{titlepage}




% --- Tabla de contenidos ---
\section*{Contents}

\renewcommand{\contentsname}{}

% set counter in roman numerals
\pagenumbering{Roman}
\thispagestyle{empty}
\tableofcontents
\thispagestyle{empty}
\newpage

% make the counter begin for 1 after the contents in arabic numerals
\pagenumbering{arabic}
\setcounter{page}{1}

\section{Corn}

\subsection{Recent developments}
Corn pricing remains anchored in weather-driven yield risk, biofuel policy, logistics, and the USDA (U.S. Department of Agriculture) balance sheet. ENSO (El Niño-Southern Oscillation, this is a climate pattern that affects weather globally) dynamics and late-season precipitation are used to condition yield and quality dispersion \citep{noaa_enso_discussion}. The USDA WASDE (World Agricultural Supply and Demand Estimates) cycle is used to reset supply-demand baselines and ending-stocks paths that reprice the curve \citep{usda_wasde,usda_understanding_wasde}. The ethanol channel remains material, with roughly 40\% of U.S. corn used for biofuels, so renewable-fuel demand and crush margins are used to tilt the outlook \citep{ers_ethanol_40,ers_ethanol_2030}. Recent commentary has emphasized a very large 2025 U.S. harvest, which is used to weigh on deferred contracts such as December 2025 (ZCZ5), all else equal \citep{reuters_record_crop_2025,ers_feedgrains_outlook}. Transport conditions and barge drafts at the Mississippi system are used to propagate regional basis and export pace \citep{ams_gtr_2023}. Storage capacity and carry incentives are used to determine the magnitude of post-harvest carries and seasonal inversions \citep{ncga_storage_2025}.

\subsection{Spot \& futures}
Cash corn is quoted in \emph{cents per bushel} at specific locations and grades; it is used to value immediate physical transactions and inventories. CBOT (Chicago Board of Trade) corn futures standardize risk transfer over horizon \(T\) with contract size \(5{,}000\) bu, tick \(1/4\) cent (\$12.50/contract), and delivery rules \citep{barchart_zc_specs}. The cost-of-carry relation is used:
\[
F=S\,e^{(r+u-y)T},
\]
where \(r\) denotes USD funding, \(u\) storage/insurance and handling, and \(y\) the convenience (inventory) yield. In storable ags, seasonal \(y\) peaks around planting/harvest uncertainty and at logistics bottlenecks; abundant storage and financing are used to steepen carries pre- and post-harvest. Location and quality basis between Illinois cash, barge/Gulf, and CBOT deliverables is used to explain deviations between spot realizations and futures marks \citep{ams_gtr_2023}.

\subsection{Mexico-linked implications}

\paragraph{Structural context and policy regime.}
Mexico is structurally short yellow corn and relies on U.S. supply, while white corn underpins tortilla consumption. Import protocols have been shaped by the 2023 biotech decree and the ensuing USMCA (United States-Mexico-Canada Agreement) dispute; the 2024 panel outcome and subsequent adjustments continue to govern allowable uses and testing regimes \citep{fas_mexico_decree_2023,ustr_usmca_biotech_2023,ustr_usmca_biotech_win_2024,reuters_mexico_gm_ban_2025,fas_mexico_grain_annual_2025}. This policy layer adds non-price variance to procurement and basis, especially near contract roll and harvest windows.

\paragraph{Transmission channels into domestic costs.}
Three levers dominate the pass-through from CBOT to Mexican delivered prices:
(i) the global level set by USDA balances and WASDE revisions \citep{usda_wasde,ers_feedgrains_outlook};
(ii) the \emph{location basis} driven by U.S. interior-to-Gulf logistics, barge drafts, and export pace \citep{ams_gtr_2023};
(iii) the USD/MXN exchange rate. Seasonal logistics constraints and barge costs widen basis precisely when Mexican buyers are most active, while FX swings amplify or cushion the result. Ethanol demand adds an endogenous pull on U.S. usage; stronger crush margins tighten balances and lift deferreds \citep{ers_ethanol_40}.

\paragraph{Operational playbook for importers and food/feed users.}
\begin{enumerate}
  \item \textit{Separate level risk from curve risk.} Price level (\(S\)) and term structure (carries, inversions) are distinct. Coverage ratios should be tied to WASDE event risk and ENSO/weather windows \citep{noaa_enso_discussion,usda_wasde}.
  \item \textit{Exploit seasonality in the curve.} Pre-/post-harvest \emph{contango} (\(r+u>y\)) is used to ladder deferred purchases; harvest \emph{inversions} are used to time physical lifts when logistics permit. Calendar spreads (e.g., Z/H, H/K) translate a view on stocks and barge capacity into hedge P\&L.
  \item \textit{Manage basis explicitly.} Define and monitor a \emph{CBOT-to-delivered Mexico} basis (Illinois/Gulf/rail) and set variance limits. Use OTC basis swaps or physical forward differentials where available to reduce residual basis risk \citep{ams_gtr_2023}.
  \item \textit{Pair with FX overlays.} Combine ZC futures (or swaps) with USD/MXN forwards/options to stabilize MXN-denominated unit costs; treat FX and corn greeks jointly in risk limits.
  \item \textit{Use options around event risk.} Ahead of WASDE or weather inflections, collars or call spreads limit upside exposure without overcommitting to volume. Position size is benchmarked to historical move distributions from \citep{usda_wasde}.
\end{enumerate}

\paragraph{Scenario guidance.}
\begin{itemize}
  \item \textit{Large U.S. crop, robust logistics.} Deferred contango widens; Mexico layers coverage out the curve and budgets carry as a known cost \citep{ers_feedgrains_outlook,reuters_record_crop_2025}.
  \item \textit{Harvest bottlenecks or low river stages.} Nearby inversions emerge; basis to Gulf widens. Mexico prioritizes near-coverage and barge/rail optionality; basis hedges are activated \citep{ams_gtr_2023}.
  \item \textit{Biotech-policy friction.} Testing/permit delays raise non-price costs and timing risk. Procurement staggers imports across origins/uses consistent with the decree and USMCA guidance \citep{fas_mexico_decree_2023,ustr_usmca_biotech_win_2024,fas_mexico_grain_annual_2025}.
  \item \textit{Ethanol-led pull.} Strong ethanol margins lift domestic U.S. usage; Mexico advances coverage in the front and reduces reliance on the back of the curve \citep{ers_ethanol_40}.
\end{itemize}

\paragraph{Policy and infrastructure implications.}
Stable, transparent import protocols reduce policy-induced basis variance; logistics investments that lower \(u\) (handling/storage) and improve corridor reliability compress delivered volatility. Public guidance that reports procurement coverage, basis benchmarks, and hedge governance improves price discovery and reduces funding costs across the chain.


\subsection{Futures term structure (12 Sep 2025)}

% cite the reference from the appendix
See Figure~\ref{fig:corn_settlements} in Appendix~A for the ZC futures strip on 12 Sep 2025.

Corn quotes are in \textit{¢/bu}. Format \texttt{430'0} \(=\) \textit{430.00 ¢/bu}. Tick \texttt{'2} \(=\) \textit{0.25 ¢} \(=\) \textit{\$12.50} per 5,000 bu. Thus, \textit{1 ¢ = \$50} per contract \citep{barchart_zc_specs}.


Up to mid-2026, settles are used to rise with maturity (\emph{contango}); around harvest months, a \emph{kink} is observed:
\begin{itemize}
  \item \textit{Pre-harvest carries:} Z25 \(\to\) H26 \(+\)17'2 \(=\) 17.25 ¢ (\(\approx\$862.50\) per contract) over \(\sim\)3 months \(\Rightarrow\) carry \(\approx 17.25/430\approx 4.0\%\) for the period (\(\sim\)16\% p.a.).
  \item \textit{Harvest kink:} N26 \(\to\) U26 \(-\)3'6 \(=\) \(-\)3.75 ¢. New-crop supply and logistics are used to compress carries or invert the nearby spread.
  \item \textit{Re-emergent carry:} U26 \(\to\) Z26 \(+\)9'4 \(=\) 9.50 ¢ as grain moves into storage and financing dominates.
\end{itemize}
Pre- and post-harvest contango indicates \(r+u>y\). Local inversions around harvest are used to signal elevated \(y\) due to supply timing, drying/quality risk, and barge or rail constraints \citep{ams_gtr_2023,ncga_storage_2025}.

\subsection{Interpretation of \texorpdfstring{ZCZ5}{ZCZ5} (one-day read)}

The one-day pricing decomposition for ZCZ5 is shown in Figure~\ref{fig:corn_one_day}.

\begin{figure}[h]
  \centering
  \includegraphics[width=0.5\textwidth]{figures/corn2.png}
  \caption{One-day pricing decomposition for ZCZ5 on 12 Sep 2025.}
  \label{fig:corn_one_day}
\end{figure}

A 91-day horizon (ACT/360) is used with \(S=\) \textit{430.00 ¢/bu} (Illinois cash), \(F=\) \textit{428.00 ¢/bu} (ZCZ5), and \(r=\) \textit{4.41\%}. The no-carry fair is obtained as
\[
T=\tfrac{91}{360},\quad F^{*}=S\,e^{rT}=430\,e^{0.0441T}\approx {434.72}\ \text{¢/bu}.
\]
A difference of \({-6.72}\) ¢/bu is observed \((F-F^{*})\) \(=\) \textit{26.9 ticks} \(=\) \textit{\$336} per contract. The spot-futures return is \(F/S-1={-0.47\%}\).

For interpretation, the carry relation \(F=S\,e^{(r+u-y)T}\) is used. The market carry is
\[
c_{\text{mkt}}=\frac{1}{T}\ln\!\frac{F}{S}
=\frac{1}{91/360}\ln\!\frac{428}{430}\approx {-1.84\%}\ \text{p.a.},
\]
so the net convenience yield is
\[
y-u \approx r-c_{\text{mkt}}\approx 4.41\%-(-1.84\%)={6.25\%}\ \text{p.a.}
\]
This \emph{negative basis} and \(F<S e^{rT}\) are consistent with \textit{backwardation}: near-dated corn is priced below pure financing carry. Operationally, inventory optionality, transport frictions, and location premia are used to raise \(y\) relative to \(r+u\). The deviation is interpreted as an inventory/logistics signal—\emph{not} an arbitrage—given storage, barge capacity, and delivery-grade constraints \citep{ams_gtr_2023,ncga_storage_2025}.


Large-crop expectations weigh on deferreds \citep{reuters_record_crop_2025}, but pre-harvest and harvest-adjacent tightness can keep nearby \(y\) elevated. WASDE revisions are used to shift the curve level; barge costs and export pace are used to move location basis; and ethanol margins are used to condition domestic offtake \citep{usda_wasde,ams_gtr_2023,ers_ethanol_40}.

\medskip
\noindent Directional bulls are typically routed to near months for higher spot beta, with curve views expressed via flatteners if stocks are expected to tighten. Producers are used to forward-sell deferred maturities when carries are rich; users are used to layer coverage into harvest dips. Procurement desks are used to pair corn hedges with USD/MXN overlays to stabilize local-currency costs, given Mexico's structural short and policy-sensitive import channel \citep{fas_mexico_grain_annual_2025}.


\subsection{Does this benefit Mexico? What should be done}


Given Mexico's structural short in yellow corn, the observed configuration—\emph{cash Illinois} at 430.00 ¢/bu, \emph{ZCZ5} at 428.00 ¢/bu, and a \emph{negative} \((F-F^{*})=-6.72\) ¢/bu—reduces \emph{hedgeable} procurement costs at the margin. One-day backwardation (\(F<S e^{rT}\)) implies a positive roll yield for long futures into expiry, while the term structure on 12 Sep 2025 displays pre-/post-harvest \emph{contango} with a \emph{harvest kink}. Net effect: near-term cover benefits from backwardation, but deferred cover faces carry (contango) that must be budgeted. The benefit materializes only if \emph{basis} (CBOT-to-delivered Mexico) and \emph{FX} (USD/MXN) risks are actively controlled.

\paragraph{Operational guidance for Mexican buyers (feed, food, processors).}
\begin{enumerate}
  \item \textit{Layered coverage with curve discipline.} Use a front-weighted hedge ladder in Z/H/K, increasing coverage into backwardation windows; reduce pace when carries widen. Tie hedge adds to WASDE and logistics checkpoints.
  \item \textit{Exploit roll when favorable.} In backwardation, long ZC futures earn positive roll yield as contracts converge; keep tenors short and roll frequently. In contango, lengthen physical tenor and hedge less of the far strip to avoid paying excessive carry.
  \item \textit{Manage basis explicitly.} Track Illinois/Gulf/rail-to-Mexico basis and set variance limits. Where available, use basis swaps or physical differentials to fix location risk before fixing futures size.
  \item \textit{Pair with FX overlays.} Hedge USD/MXN for the same notional and tenor as corn exposure; treat corn and FX jointly in VaR and budget. Use forwards for base cover and options for event tails.
  \item \textit{Use options around event risk.} Ahead of WASDE or weather inflections, deploy collars or call spreads sized to historical move distributions; avoid over-hedging volume before logistics windows are secured.
\end{enumerate}

\paragraph{Investor playbook (policy-neutral).}
\begin{itemize}
  \item \textit{Directional view:} Express with near contracts for higher spot beta; calibrate size to weekly vol and WASDE dates.
  \item \textit{Curve view:} If there was expected stocks to tighten or funding to ease, run \emph{flatteners} (long near/short far); if ample stocks/logistics, \emph{steepeners}.
  \item \textit{Carry harvesting:} Systematic long-only carry works in backwardation; in contango, restrict tenors, or finance long futures against short calendar spreads.
\end{itemize}

\paragraph{Policy and infrastructure priorities for Mexico.}
\begin{enumerate}
  \item \textit{Reduce logistics/storage frictions (\(u\)).} Improve port, rail, and storage capacity to compress delivered basis and lower carry; prioritize harvest-import corridors.
  \item \textit{Regulatory clarity.} Maintain transparent, predictable biotech/testing protocols to cut non-price variance at customs and reduce timing premia.
  \item \textit{Hedging governance.} Promote standardized hedge frameworks (corn + FX) for public entities and SMEs; ensure tax/royalty neutrality between spot and hedged outcomes.
  \item \textit{Market transparency.} Publish procurement coverage, basis benchmarks, and hedge cadence to lower financing spreads and anchor expectations along the supply chain.
\end{enumerate}

\noindent
The present mix—spot strength with localized backwardation at the front and carries farther out—can \emph{benefit} Mexico's import bill if basis and FX are actively hedged and logistics are reliable. Tactical gains come from timing cover into backwardation and limiting exposure to expensive carries; structural gains come from policies that compress \(u\) and stabilize the import regime.


\section{Crude Oil}

\subsection{Recent developments}
Crude oil remains anchored by the classical triad of fundamentals, policy coordination, and dollar conditions. Global demand growth and inventory paths continue to govern the balance; OPEC+ supply management modulates prompt availability; and the USD transmits monetary conditions into non-USD purchasing power \citep{eia_prices_2023,eia_opec_2024,bis_usd_commodity_2023,ecb_oil_usd_2024}. Into mid-September 2025, front-WTI traded in the low \$60s per barrel while carry signals pointed to persistent, though not extreme, prompt tightness.

\subsection{Spot \& futures}
WTI spot reflects contemporaneous physical scarcity and the USD level; by contrast, the futures curve prices the intertemporal trade-off between holding inventory and deferring delivery. The standard cost-of-carry relation is used:
\[
F_T=S_0\,e^{(r+u-y)T},
\]
with \(r\) the USD funding rate, \(u\) storage/insurance, and \(y\) the convenience (lease) yield. Backwardation (\(F_T<S_0e^{rT}\)) occurs when \(y>r+u\); contango when \(y<r+u\). Contract units and expiries follow CME specifications for WTI (1{,}000 bbl, USD/bbl) \citep{cme_cl_specs,cme_cl_calendar}. Risk-neutral “no-carry” fair values in this report use \(F^{*}=S_0 e^{rT}\) with SOFR-based \(r\) \citep{frbny_sofr}.

\subsection{Mexico-linked implications}
For Mexico, oil levels and curve shape transmit through fiscal and external accounts. Higher spot improves upstream realized prices and state revenues; pronounced backwardation raises the value of prompt barrels relative to deferrals, but compresses the forward cover available to hedge future cash flows. Conversely, contango lowers prompt realizations but cheapens deferred hedges. Given Pemex's financing and investment plans, a stable, shallow backwardation is operationally preferable to extreme tightness: it supports near-term cash margins while avoiding destabilizing roll dynamics \citep{reuters_oil_peak_2008,worldbank_oil_spike_2008}. Policy levers that reduce logistics and storage frictions effectively lower \(u\) and stabilize basis, improving the translation from global prices to domestic income.

\paragraph{Transmission channels.}
Three first-order channels transmit crude dynamics into Mexico's macro-financial stance: (i) the \emph{fiscal} channel via upstream realized prices, production volumes \(Q_t\), and the Maya/WTI differential; (ii) the \emph{external} channel through the hydrocarbon trade balance (crude exports vs.\ refined-product imports); and (iii) the \emph{financial} channel via USD/MXN, sovereign risk premia, and the equity cost of capital for energy-linked corporates. Formally, upstream cash margin is \(M_t \approx Q_t\,[S_t-\delta_t]-\text{OPEX}_t\), where \(S_t\) is the benchmark (WTI/Brent proxy), \(\delta_t\) the quality/location discount (e.g., Maya to WTI), and \(\text{OPEX}_t\) operating costs. Fiscal intake and external balances scale with \(M_t\) and with the sign of refined-product net imports.

\paragraph{Role of curve shape for cash-flow timing.}
With \emph{orderly backwardation} and modest day-to-day parallel level moves, prompt barrels are valued above deferred. This configuration is advantageous for near-term monetization but reduces the forward price available for long-dated hedges. In accounting terms, a government or SOE hedger that sells \(n\) futures across a ladder \(\{T_i\}\) locks
\[
\mathbb{E}[\text{Revenue}] \approx \sum_i Q_{T_i}\,F_{T_i},
\]
so backwardation (\(F_{T_i}<S_0 e^{rT_i}\)) lowers hedge strikes even as it supplies \emph{positive roll} to long consumers. Conversely, contango would raise forward strikes but erode consumer roll. The week's diagnostics show stable \(y-u\) at short horizons; hedge design can therefore prioritize \emph{tenor diversification} rather than chasing transient slope noise.

\paragraph{Logistics, storage, and basis.}
Domestic logistics and storage conditions map into the effective \(u\) and into location basis \(\delta_t\). Lower pipeline/terminal frictions reduce \(u\) and compress \(\delta_t\), improving pass-through from global benchmarks to domestic realizations. In backwardation, low working inventories are rational; however, excessively lean stocks amplify downside tail risk if \emph{y} spikes. A policy mix that keeps minimum operational inventories while upgrading storage flexibility reduces the volatility of \(M_t\) without materially sacrificing carry.

\paragraph{FX and monetary-policy interactions.}
Higher crude supports the terms of trade and, conditional on global risk, can be MXN-supportive. Yet FX pass-through to domestic fuel prices affects headline inflation. If policy opts for partial smoothing (implicit fuel-price stabilization), fiscal buffers must be pre-committed or hedged. Otherwise, inflation volatility tightens the constraint set for monetary policy. Practically, energy-linked issuers should pair oil hedges with FX overlays to stabilize MXN cash flows; otherwise, USD price gains can be offset by MXN appreciation.

\paragraph{Corporate treasury and investor implementation.}
For upstream exposure, a \emph{laddered short} in the liquid front two or three contracts balances execution depth and slope risk; the documented stability of short-dated \(y-u\) argues for systematic rebalancing rather than discretionary timing. For refiners/marketers, \emph{long prompt / short deferred} structures monetize positive roll in backwardation while capping input-cost spikes. Cross-asset investors can express macro views via \emph{curve trades}: expected inventory rebuilds or lower USD rates support flatteners (long near/short far); persistent OPEC+ discipline or logistics constraints support steepeners in backwardation.

\paragraph{Policy and operating priorities.}
(i) \emph{Lower \(u\)}: invest in storage reliability and throughput to reduce effective storage/operational costs and to smooth basis.  
(ii) \emph{Standardize hedging governance}: publish a clear tenor ladder and risk limits; disclose hedge strikes and coverage ratios to anchor funding costs.  
(iii) \emph{Integrate FX and commodity risk}: mandate MXN cash-flow stabilization targets with coordinated oil-FX overlays.  
(iv) \emph{Transparency on quality differentials}: regular disclosure of the Maya/WTI (or delivered) differential \(\delta_t\) improves revenue forecasting and reduces uncertainty premia.


Given spot in the low \$60s and shallow, stable backwardation, near-term upstream margins are supported while roll dynamics remain manageable. The configuration benefits Mexico if logistics keep \(u\) contained and if hedging policy converts favorable prompt economics into stable, multi-quarter MXN cash flows. The principal risks are a shock to inventories (deepening backwardation and basis) or a USD-driven tightening in global financial conditions; both warrant pre-defined hedge and liquidity triggers.


\subsection{Futures term structure (12 Sep 2025)}

See Figure~\ref{fig:crudeoil_settlements} in Appendix~A for the CL futures strip on 12 Sep 2025.

\textit{Shape and slopes.} The strip exhibits \emph{orderly backwardation}:
\[
\text{Oct-25 } {62.69} \;\rightarrow\; \text{Dec-25 } {62.19}
\;\rightarrow\; \text{Mar-26 } {61.95}
\;\rightarrow\; \text{Dec-26 } {61.77}
\;\rightarrow\; \text{Oct-27 } {62.08}.
\]
The 1-year slope (Oct-25\(\to\)Oct-26) is \(61.74/62.69-1\approx {-1.52\%}\) p.a., implying \(r+u-y\approx -1.5\%\) p.a. (therefore \(y>r+u\)). A slight re-steepening at the very back is consistent with normalization of tightness beyond one year.

\textit{Daily move.} A near-parallel \(\,+0.32\) to \(+0.36\) USD/bbl shift is observed across the strip on the session, read as a \emph{spot-led} rally with curve shape broadly preserved—i.e., news raised the level, not the intertemporal premium.

\textit{Carry and rolls.} Calendar spreads such as Oct/Nov \(+0.27\), Nov/Dec \(+0.23\), and Oct/Dec \(+0.50\) encode positive roll income for long-only consumers and negative roll for producer shorts in the current backwardation regime.

\subsection{Interpretation of \texorpdfstring{CLV25}{CLV25} (one-day read)}

Using \(S_0=\$62.69\), \(r=4.41\%\) (ACT/360), and \(T=10/360\), the no-carry fair is
\[
F^{*}=S_0 e^{rT} \approx 62.69\,e^{0.0441\cdot(10/360)}={62.765}.
\]

\begin{figure}[h]
  \centering
  \includegraphics[width=0.5\textwidth]{figures/wti.png}
  \caption{One-day pricing decomposition for CLV25 on 12 Sep 2025.}
\end{figure}

The observed settlement \(F={62.69}\) implies \(F<F^{*}\) by \(\$0.075\) (backwardation). Two equivalent diagnostics are used:

\(\bullet\) \emph{Market carry vs.\ spot:} \(c_{\text{mkt}}=\frac{1}{T}\ln(F/S_0)\approx 0\) p.a., since \(F\approx S_0\).

\(\bullet\) \emph{Net convenience minus storage:} from \(F/F^{*}=e^{(u-y)T}\),
\[
y-u=\frac{1}{T}\ln\!\Big(\frac{F^{*}}{F}\Big)\approx {4.32\%}\ \text{p.a.}
\]
Numerically, \(y-u\) is close to \(r\), indicating that the discount to \(F^{*}\) is largely explained by convenience yield offsetting financing (with storage \(u\) small at the 10-day horizon). This is the canonical signature of \emph{mild but persistent prompt tightness} \citep{eia_backwardation_2013,milonas_convenience_2024}.

\subsection{Weekly term-structure diagnostics (5-12 Sep 2025)}

\begin{figure}[h]
  \centering
  \includegraphics[width=1\textwidth]{figures/crude_oil_pricing_over_the_week.pdf}
  \caption{Crude oil pricing decomposition over the week for ZCZ5.}
\end{figure}

\textit{Data.} SOFR ranged \(4.39\)-\(4.42\%\); horizons declined from 17 to 10 days. Spot and front futures were essentially equal each day; the theoretical no-carry fair \(F^{*}\) exceeded market by \(\$0.075\)-\(\$0.127\).

\begin{figure}[h]
  \centering
  \includegraphics[width=0.7\textwidth]{figures/crude_oil_difference.pdf}
  \caption{Difference between market and no-carry fair over the week for ZCZ5.}
\end{figure}

\textit{Stable backwardation intensity.} Although \((F-F^{*})\) becomes less negative through the week, this is \emph{mechanical time decay}—as \(T\downarrow\), the present-value gap from financing shrinks. The carry metric that nets out horizon effects,
\[
y-u=\frac{1}{T}\ln\!\Big(\frac{F^{*}}{F}\Big),
\]
is remarkably stable at \({4.30\%{-}4.33\%}\) p.a. each day. Hence, the convenience yield continues to offset funding with little drift.

\textit{Spot-led level variation, curve preserved.} Day-to-day price changes are dominated by spot (macro/USD) rather than by shifts in \((r+u-y)\). This matches the session commentary above: level up-moves occur with an approximately unchanged slope, consistent with fundamentals/news raising the spot anchor while physical tightness stays moderate \citep{eia_prices_2023,eia_opec_2024,bis_usd_commodity_2023}.

\textit{Micro implications.} For longs, backwardation provides positive roll yield into expiry; for producer shorts, rolls are a headwind but hedge effectiveness is high because \(y-u\) is stable. The absence of sharp slope changes reduces basis risk for short-dated hedges.

\subsection{Does this benefit Mexico? What should be done}
It benefits Mexico by providing a stable revenue stream from oil exports. The configuration—spot in the low \$60s with steady, shallow backwardation—supports upstream cash flow and state revenue without imposing extreme roll costs or volatility.

\paragraph{Investor actions (policy-neutral).}
\begin{enumerate}
  \item \emph{Decompose risks.} Manage level risk (spot) and term risk (roll/carry) separately. Use short-dated futures for beta; use calendar spreads to express views on slope normalization.
  \item \emph{Hedge structure.} Producer hedges can be laddered in the front two contracts to exploit stable backwardation; consumers can pre-buy prompt barrels and keep deferred cover lighter to benefit from positive roll.
  \item \emph{Trigger mapping.} Monitor OPEC+ guidance and U.S. rate moves: a durable fall in \(r\) or inventory builds should flatten the curve; tightening logistics raises \(y\) and deepens backwardation.
\end{enumerate}

\paragraph{Policy/operating recommendations for Mexico.}
\begin{enumerate}
  \item \emph{Lower logistics/storage frictions.} Reducing \(u\) via infrastructure and operational reliability improves realized margins and stabilizes basis.
  \item \emph{Hedge governance.} Standardize hedge programs that couple oil price hedges with FX overlays to dampen peso-revenue volatility.
  \item \emph{Transparency and cadence.} Publish regular hedge/carry metrics (e.g., inferred \(y-u\)) alongside production guidance to anchor market expectations and lower funding costs.
\end{enumerate}



\section{Silver}

\subsection{Economic role and macro-financial channels}
Silver is a dual-use metal with a large industrial footprint (electronics, photovoltaics, medical and chemical applications) and a non-trivial investment component. As a result, its pricing is shaped by both end-use demand and financial conditions. Cyclical expansions raise manufacturing throughput and technology adoption, lifting physical offtake; global slowdowns reverse that impulse. Monetary policy and term premia transmit through discount rates and the USD: a stronger dollar tightens global financial conditions and lowers non-USD purchasing power, dampening demand for dollar-priced commodities; a softer USD does the opposite \citep{silver_institute_wss_2024,usgs_silver_mcs_2024,bis_usd_commodity_2023}.

\subsection{Policy and geopolitical drivers}
Trade restrictions, tariffs, sanctions, and logistics frictions reallocate flows across refining hubs and end-user markets, altering local basis and inventory dynamics; these mechanisms operate alongside the macro channels above \citep{silver_institute_wss_2024}.

\subsection{Historical stress episode}
The 1980 episode (“Silver Thursday”) illustrates how leverage, concentrated positioning, and margining can dominate price discovery. After a speculative run-up to \$49.45/oz, rule changes and margin calls precipitated a sharp decline on March 27, 1980, with broader market spillovers \citep{britannica_silver_thursday,nyt_1980_silver_thursday}.

\subsection{Recent developments}
In early September 2025, spot silver traded above \$40/oz and marked 14-year highs amid rising Fed-easing probabilities and a weaker USD; research commentary highlighted ETF inflows and a persistent physical deficit \citep{reuters_silver_14y_sep1,reuters_tradingday_sep11,reuters_anz_raises_silver,lbma_q2_2025}. Corporate activity remains supportive for Mexico-centric supply, including consolidation around the Juanicipio asset \citep{reuters_paasmag}. Spot benchmarking relies on the LBMA Silver Price, while COMEX futures provide standardized exposure and hedging along the term structure \citep{lbma_prices,cme_silver_overview}. 
Mexico is the world's largest silver producer, so revenue, royalties, and FX inflows are sensitive to the level and volatility of silver. Higher prices improve internal cash generation and capex flexibility; conversely, regulatory or logistical frictions widen local basis and delay monetization. Through the terms-of-trade channel, stronger silver can be peso-supportive when broader macro conditions are aligned \citep{reuters_mx_top_silver}.

\subsection{Spot \& futures}
The spot price refers to unallocated silver in London (or a deliverable loco) and is used to value immediate transactions and inventories. COMEX futures internalize financing and inventory economics over horizon \(T\) via the cost-of-carry relation
\[
F=S\,e^{(r+u-l)T},
\]
with \(r\) the USD funding rate, \(u\) storage/insurance, and \(l\) the lease (convenience) rate. Tight nearby physical conditions raise \(l\) and can generate backwardation \((F<S\,e^{rT})\); abundant storage and higher financing tilt the curve toward contango. Thus, spot primarily reflects contemporaneous scarcity and dollar conditions, while futures reflect the intertemporal trade-off between owning inventory today versus financing delivery later.

The \emph{spot} benchmark is the LBMA Silver Price (or XAG/USD quotes for unallocated metal in London); it is used to value immediate physical transactions, inventory revaluation, and cash-market settlements \citep{lbma_prices}. The COMEX futures curve is used to transfer price risk across time and to standardize hedging and speculative exposure; contract design, margining, and delivery mechanics shape its microstructure \citep{cme_silver_overview}. In practice, three news-sensitive wedges separate spot from futures: (i) the \emph{financing-storage-lease} term in the cost-of-carry \(F=S\,e^{(r+u-l)T}\), (ii) \emph{location/quality} basis between London unallocated and COMEX-deliverable bars, and (iii) \emph{timing} basis linked to near-term ETF creations/redemptions and refinery/warehouse flows. When easing U.S. rate expectations strengthen non-USD demand and ETF creations accelerate, it is common that spot tightens faster than the curve (lease rate \(l\uparrow\)), generating transient backwardation; as inventories are replenished and funding/storage regain prominence, slight contango reappears. Hence, spot is used as a barometer of contemporaneous scarcity and USD conditions, whereas futures are used to price the intertemporal trade-off between carrying inventory and deferring settlement.

\subsection{Mexico-linked implications}
Near-term shipments monetize elevated spot first; medium-dated cash flows depend on the curve's carry and roll costs. Hedge design should distinguish (i) price level risk (spot) from (ii) term-structure risk (roll yield and basis), recognizing that cross-currency and policy shocks can modulate both even when the global price is strong.

Mexico's role as the world's largest silver producer means that domestic cash flows, royalties, and FX receipts are highly exposed to these spot-futures dynamics \citep{reuters_mx_top_silver}. The following news-driven scenarios are operationally relevant:
\begin{enumerate}
  \item \textit{ETF inflow surges and USD weakness.} Spot premia rise first; it is used by producers with near-term shipments to monetize higher LBMA realizations. If COMEX contango persists, rolling short futures hedges entails modest carry costs; if backwardation appears, rolls can be revenue-positive.
  \item \textit{Refinery or logistics bottlenecks.} Delays in moving Mexico-origin doré or concentrates to refiners widen the location basis versus London. It is used to hedge with COMEX futures, but performance depends on the basis path; treasury policy should stress-test basis risk alongside price risk.
  \item \textit{Funding-cost swings.} Shifts in dollar funding alter \(r\) and, with stable lease \(l\), move the curve even if spot is unchanged; this is used to reassess hedge tenors because forward premia/discounts change while realized sales prices do not.
  \item \textit{Domestic policy shocks.} Changes in royalties, permitting, power tariffs, or labor conditions affect mining margins and supply timing; it is used to adjust production hedges and to reassess the mix of spot sales versus forward coverage given altered cash-need profiles.
  \item \textit{FX interaction (USD/MXN).} Stronger silver improves Mexico's terms of trade and can support MXN conditionally on global risk; however, a stronger MXN reduces peso revenues for unhedged USD sales. It is used to pair metal hedges with FX hedges to stabilize MXN cash flows.
\end{enumerate}
Elevated spot levels benefit near-term Mexican exports directly, while the sign and size of the futures basis determine carry and roll outcomes on hedges. Policy- and logistics-sensitive basis risk is material; it is used to complement price hedging with explicit basis and FX risk management, recognizing that the curve reflects financing and inventory conditions that need not move one-for-one with spot.



\subsection{Futures term structure (12 Sep 2025)}

See Figure~\ref{fig:silver_settlements} in Appendix~A for the SI futures strip on 12 Sep 2025.

The silver curve on \emph{12 Sep 2025} is interpreted through the cost-of-carry relation
\[
F_T = S_0\,e^{(r+u-l)T},
\]
where \(r\) is USD funding, \(u\) storage/insurance, and \(l\) the lease (convenience) rate. Curve \emph{shape} identifies the sign and size of \((r+u-l)\); parallel \emph{level} moves are used to diagnose spot-led shocks, while \emph{non-parallel} moves are used to diagnose changes in carry/inventory conditions.



A clear \emph{contango} is observed across listed maturities:
\[
\text{SEP-25 } {42.387}\ \rightarrow\ \text{DEC-25 } {42.830}\ \rightarrow\ 
\]
\[
\text{MAR-26 } {43.322}\ \rightarrow\ \text{DEC-26 } {44.682}\ \rightarrow\ 
\]
\[
\text{MAR-27 } {45.089}\ \rightarrow\ \text{SEP-27 } {45.877}.
\]
The one-year slope (DEC-25 \(\rightarrow\) DEC-26) is
\[
\frac{44.682}{42.830}-1 \approx {4.32\%}\ \text{p.a.},
\]
and the two-year slope (SEP-25 \(\rightarrow\) SEP-27) is
\[
\Big(\frac{45.877}{42.387}\Big)^{1/2}-1 \approx {4.04\%}\ \text{p.a.}
\]
These slopes are used as reduced-form estimates of \((r+u-l)\) over the corresponding horizons, implying \(r+u-l\gtrsim 4\%\) p.a.; inventories are not tight enough to invert the curve (\(l\) does not dominate \(r+u\)).

A near-parallel increase of about \({+0.68}\) USD/oz along the strip (\(\sim {+1.6\%}\) at the front) is observed. Parallelism is used to attribute the move primarily to a \emph{spot-led shock} (e.g., dollar softness and investor demand), while the persistence of contango indicates that intertemporal premia \((r+u-l)\) are broadly unchanged on the day. This diagnosis is consistent with the one-day SIZ25 read, where slight contango is obtained despite elevated spot.


Open interest and volume concentrate in \textit{DEC-25} (vol \(\approx 72{,}263\), OI \(\approx 133{,}690\)); \textit{MAR-26} is the secondary node (vol \(\approx 2{,}265\), OI \(\approx 16{,}343\)). It is used to execute hedges and curve views primarily in these nodes to minimize execution and basis risk. Quotes in thin months are often marking references rather than active trading venues.


Calendar spreads summarize carry costs and roll P\&L:
\[
\text{DEC-25/MAR-26} \approx +0.492,\qquad \text{DEC-25/DEC-26} \approx +1.852.
\]
In contango, long futures positions face \emph{negative roll yield} if spot is unchanged (convergence down toward spot); short positions benefit from carry. This mapping is used in hedge design (e.g., producer shorts) and in curve trades (e.g., flatteners/steepeners).


Three levers are used to connect macro news to curve diagnostics: (i) \emph{rates}—lower USD rates compress \((r+u-l)\) and flatten contango; (ii) \emph{inventory/lease}—tight physical markets raise \(l\), flattening or inverting the curve; (iii) \emph{storage/operations}—changes in storage and insurance costs shift \(u\), steepening or flattening depending on direction. The observed contango with a parallel up-move is, therefore, read as “risk-on/spot-led, inventory not binding.”


Directional bulls are typically routed to near maturities for higher spot beta and may hedge roll via calendar spreads (e.g., long DEC-25 / short MAR-26) if carry is expected to compress. Industrial users hedging inputs are advised to lock deferred maturities knowingly that the quoted contango is the carrying cost they pay. Curve views are implemented as \emph{flatteners} if easing USD rates or tighter inventories are expected; otherwise, \emph{steepeners} are preferred.


When spot \(S_0\), USD term rates, and storage estimates are available, the convenience/lease component can be inferred as
\[
y \approx r + u - \frac{1}{T} \ln\!\Big(\frac{F_T}{S_0}\Big),
\]
and its time-variation is used as an early warning of physical tightness (rising \(y\) with flat \(r\)) or of storage/funding pressure (rising \(u\) or \(r\) with flat \(y\)).

\medskip
The term structure on 12 Sep 2025 exhibits orderly contango with a spot-led parallel rally. It is used to conclude that intertemporal premia remain positive but contained, consistent with strong spot demand and adequate deliverable inventories. This conclusion coheres with the one-day SIZ25 basis and the week's transition from mild backwardation to mild contango, linking news flow to a coherent curve narrative.


\subsection{Interpretation of \texorpdfstring{SIZ25}{SIZ25} (one-day read)}
A \emph{positive basis} is observed: \(F=42.83>F^{*}=42.74\), a premium of \$0.09/oz (18 ticks, \$450 per 5{,}000-oz contract). With \(S=42.195\) and \(T=108/360\), the market carry is
\[
c_{\mathrm{mkt}}=\frac{1}{T}\ln\!\frac{F}{S}
=\frac{1}{0.30}\ln\!\frac{42.83}{42.195}\approx 4.97\%\ \text{p.a.}
\]
Relative to \(r=4.41\%\), the implied net storage minus lease is
\[
u-l=c_{\mathrm{mkt}}-r\approx 0.6\%\ \text{p.a.},
\]
consistent with \emph{slight contango}: financing plus storage marginally exceed the lease rate, so the future sits above spot and above the pure-financing benchmark \(F^{*}=S\,e^{rT}\). This aligns with contemporaneous news: strong demand and a softer USD lift spot, yet lease rates have not risen enough to overcome carry costs over 108 days.

\begin{figure}[h]
\centering
\includegraphics[width=0.7\textwidth]{figures/silver_pricing_one_day.png}
\caption{One-day pricing decomposition for SIZ25 on 12 Sep 2025.}
\label{fig:silver_one_day}
\end{figure}


A small, but statistically meaningful, \emph{positive basis} is observed on \emph{September 12, 2025}: \(F=42.83>F^{*}=42.74\), i.e., a premium of \(\$0.09/\mathrm{oz}\) (18 ticks, \(\$450\) per 5{,}000-oz contract). With \(S=42.195\) and \(T=108/360\), the market-implied carry is
\[
c_{\mathrm{mkt}}=\frac{1}{T}\ln\!\frac{F}{S}=\frac{1}{0.30}\ln\!\frac{42.83}{42.195}\approx 4.97\% \ \text{p.a.}
\]
Given a funding benchmark \(r=4.41\%\), the net inventory term is obtained as
\[
u-l=c_{\mathrm{mkt}}-r\approx 0.60\%\ \text{p.a.}
\]
This decomposition is used to interpret the premium as \emph{slight contango}: financing and storage together exceed the lease (convenience) rate by a few basis points on an annualized basis, so the future prices above both spot and the pure-financing benchmark \(F^{*}=S\,e^{rT}\).

Contemporaneous headlines emphasize (i) elevated spot levels on rising Fed-cut probabilities and a softer USD, (ii) ETF-related investor demand, and (iii) discussion of structural deficits. These elements are used to lift the \emph{subyacente} first; however, the futures price capitalizes financing and inventory over the next 108 days. The small, positive \(u-l\) indicates that, notwithstanding robust spot demand, lease rates have not surged enough to dominate funding and storage for this horizon. Adequate deliverable stocks, available storage, and year-end financing conditions can sustain a modest contango even as spot remains buoyant. Economically, this is read as a \emph{carry/liquidity configuration}, not as a dislocation or an arbitrage.
 The premium sits well within typical no-arbitrage bands once bid-ask, exchange fees, collateral haircuts, and delivery frictions are recognized. Positioning should therefore emphasize curve-shape risk (carry and roll) rather than seeking to monetize an illusory mispricing.

\subsection{Weekly term-structure diagnostics (5-12 Sep 2025)}
Table-based evidence indicates a transition from \emph{mild backwardation} early in the week (negative Market \(-\) Theoretical) to \emph{mild contango} by Thursday-Friday (positive spreads). Interpreting \(u-l=\tfrac{1}{T}\ln(F/F^{*})\) with ACT/360:

\begin{figure}[h]
\centering
\includegraphics[width=1\textwidth]{figures/silver_pricing_over_the_week.pdf}
\caption{Weekly evolution of Market \(-\) Theoretical and risk premium for SIZ25.}
\label{fig:silver_week}
\end{figure}

\begin{itemize}
  \item \textit{9/5-9/10:} negative spreads (\(-\$0.02\) to \(-\$0.11\)/oz) imply \(l>r+u\): nearby inventory scarcity and/or elevated lease rates.
  \item \textit{9/11-9/12:} positive spreads (\(+\$0.02\) to \(+\$0.09\)/oz) imply \(r+u>l\): easing tightness and/or stronger preference for futures exposure that capitalizes carry.
\end{itemize}
The Risk Premium \((F/S-1)\) remains small (≈1.06-1.50\%), dipping when spot outperforms and recovering as futures regain a modest premium. Economically, this premium summarizes carry (funding \(+\) storage \(-\) lease) and is best interpreted as a liquidity/carry effect rather than an arbitrage opportunity.

\begin{figure}[h]
\centering
\includegraphics[width=0.7\textwidth]{figures/silver_difference.pdf}
\caption{Daily difference \(F-F^{*}\) and sign (backwardation vs.\ contango).}
\label{fig:silver_difference}
\end{figure}

The term structure transitions from \emph{mild backwardation} to \emph{mild contango} over the week. Early readings (\(F-F^{*}<0\)) are used to infer \(l>r+u\) for short horizons—consistent with transient tightness or elevated borrow in the physical market—while late-week readings (\(F-F^{*}>0\)) indicate \(r+u>l\) as funding and storage reassert dominance.

Using \(u-l=\frac{1}{T}\ln(F/F^{*})\) (ACT/360; \(T\) declines from \(\sim 0.319\) to \(\sim 0.300\)):
\begin{itemize}
  \item \textit{Fri 9/5:} \(F-F^{*}=-\$0.024\) \(\Rightarrow\) mild backwardation; \(u-l\approx-0.18\%\) p.a. It is used to read near-term inventory value as slightly elevated relative to carry.
  \item \textit{Mon 9/8:} \(F-F^{*}=-\$0.011\) \(\Rightarrow\) mild backwardation; \(u-l\approx-0.08\%\) p.a. Tightness signal weakens; curve moves toward flat.
  \item \textit{Tue 9/9:} \(F-F^{*}=-\$0.111\) \(\Rightarrow\) backwardation; \(u-l\approx-0.87\%\) p.a. A sharper spot-led impulse is used to widen the lease premium.
  \item \textit{Wed 9/10:} \(F-F^{*}=-\$0.114\) \(\Rightarrow\) backwardation; \(u-l\approx-0.89\%\) p.a. Persistence of tightness; curve remains inverted at the front.
  \item \textit{Thu 9/11:} \(F-F^{*}=+\$0.017\) \(\Rightarrow\) slight contango; \(u-l\approx+0.13\%\) p.a. The balance shifts; funding/storage begin to dominate.
  \item \textit{Fri 9/12:} \(F-F^{*}=+\$0.085\) \(\Rightarrow\) contango; \(u-l\approx+0.66\%\) p.a. A normalized carry regime is used to characterize the close of the week.
\end{itemize}


Three operational channels are typically implicated: (i) \emph{funding}—a rise in Fed-cut odds lowers expected discount rates but can steepen near-term collateral demand, dynamically affecting \(r\) relative to \(l\); (ii) \emph{inventory/warehouse}—refinery and warehouse flows can replenish deliverables, compressing lease premia; (iii) \emph{term demand}—investors may prefer futures exposure to defer physical handling, lifting \(F\) relative to spot once the initial spot-led surge abates. The observed risk premium \((F/S-1\approx 1.06\% \text{ to } 1.50\%)\) remains small; it is used as a summary statistic of carry rather than a tradable edge.

\subsection{Does this benefit Mexico? What should be done?}

As the leading global producer, Mexico is positioned to benefit from elevated spot and orderly term structure. The one-day contango and the week's re-steepening are used to indicate that (i) near-term shipments monetize high spot realizations, and (ii) forward coverage can be implemented with manageable roll costs. FX pass-through matters: stronger silver improves terms of trade and is often peso-supportive, yet a stronger MXN reduces peso-denominated revenues unless currency risk is hedged.

\paragraph{Implications for investors (policy-neutral).}
\begin{enumerate}
  \item \textit{Separate spot risk from curve risk.} It is used to treat price level (\(S\)) and roll/carry (\(F-S\), shape) as distinct. Long spot exposure benefits from the current level; futures strategies should model carry explicitly.
  \item \textit{Hedge design.} For producers and MXN-based portfolios, it is used to pair metal hedges with FX hedges to stabilize peso cash flows. With slight contango, rolling short COMEX futures entails modest carry costs; in backwardation windows, rolls may be revenue-positive.
  \item \textit{Tenor selection.} It is used to choose hedge tenors where \(u-l\) is stable and liquidity is deep (e.g., front two contract months), minimizing basis and execution risk.
  \item \textit{Basis management.} Location and quality basis between London unallocated and COMEX deliverable bars should be monitored; treasury policy is used to set limits on basis drift and to pre-approve alternative hedging venues if needed.
\end{enumerate}

\paragraph{Implications for Mexico (policy and operating environment).}
\begin{enumerate}
  \item \textit{Logistics and permitting reliability.} It is used to prioritize predictable permitting timelines and logistics corridors; stable, transparent processes compress location basis and reduce lease-rate spikes tied to bottlenecks.
  \item \textit{Energy and power reliability.} Processing and refining require stable power; reliability improvements lower effective storage/operating costs \(u\), improving net margins across the cycle.
  \item \textit{Market infrastructure and risk management.} It is used to encourage the adoption of standardized hedge programs (including FX overlays) among medium-size producers to align with best practices and reduce macro-volatility transmission to local cash flows.
  \item \textit{Tax and royalty neutrality to hedging.} It is used to ensure that fiscal rules treat realized hedge outcomes neutrally relative to spot sales, avoiding distortions that discourage prudent risk transfer.
\end{enumerate}

\medskip
The current configuration—high spot with slight, later-week contango—supports Mexican revenues while keeping roll costs contained. For investors, it is used to favor disciplined hedge-and-carry implementations that respect term-structure signals. For policymakers, reducing operational frictions and basis volatility enhances the capacity to translate favorable global prices into stable domestic income and investment.



\section{IPC}

\subsection{Recent developments}
Mexico's benchmark \emph{S\&P/BMV IPC} printed fresh highs into \emph{12 Sep 2025} as global risk appetite improved on rising Fed-easing probabilities and a firm peso backdrop \citep{reuters_ipc_record_2025,reuters_usdmxn_quote}. Local conditions remain supportive but gradual: Banxico's pace of easing is measured amid stable inflation expectations, while fiscal messaging and quasi-sovereign considerations (Pemex) shape equity risk premia and multiples \citep{bloomberg_mx_inflation_2025,mnd_inflation_band_2025,reuters_budget_2025,reuters_pemex_plan_2025}. Trade-policy headlines (e.g., prospective auto tariffs) interact with the nearshoring narrative and the industrial complex, generating sectoral rotations \citep{reuters_tariffs_china_autos_2025,reuters_border_jobs_2025}.

\subsection{Spot \& futures}
The IPC is float-adjusted, market-cap weighted under published rules \citep{spdj_ipc_page,spdj_bmv_methodology}. Spot index levels reflect contemporaneous earnings, discount rates, and MXN conditions; futures on MexDer transfer price risk across time. The fair-value relation
\[
F_T = S_0\,e^{(r-q)T}
\]
is used, where \(r\) is the relevant funding rate and \(q\) the dividend yield over \([0,T]\). In practice, thin trading in deferred IPC maturities and dividend-timing uncertainty add microstructure noise around this benchmark \citep{mexder_ipc_fut}. In the accompanying tables, SOFR is used as the working \(r\) for consistency across assets; ideally an MXN funding curve would be applied for precision. This choice does not alter the qualitative reading of the week's carry dynamics.

\subsection{Mexico-linked implications}
Equity levels and futures carry transmit into financing conditions for issuers, pension/wealth portfolios, and the hedging cost of domestic asset managers. A steeper contango (\(r>q\)) raises the “cost of carry” for long futures exposure but eases the roll for short index overlays; compressing contango (via lower \(r\) or higher \(q\)) does the opposite. Policy credibility (fiscal/Pemex) and Banxico guidance compress risk premia, supporting spot while also affecting the slope through the \(r\) channel.

\paragraph{Transmission channels.}
The IPC level and its futures carry transmit to Mexico's real and financial economy through five primary channels:
\begin{enumerate}
  \item \textit{Sovereign and macro risk premia.} Credible fiscal guidance and Pemex execution compress sovereign spreads, supporting equity multiples and lowering the \emph{required} discount rate; this improves spot and, via the rate channel, can compress \(r-q\) on the strip \citep{reuters_budget_2025,reuters_pemex_plan_2025}.
  \item \textit{FX and external balance.} Pro-risk global conditions and favorable terms of trade strengthen MXN, which reduces imported inflation and supports real income, but dampens exporters' MXN revenues unless FX is hedged; IPC futures help neutralize equity beta while FX overlays manage currency co-movement \citep{reuters_usdmxn_quote}.
  \item \textit{Corporate financing and issuance windows.} Higher spot and lower equity risk premia improve primary/secondary equity issuance conditions; modest contango (\(r>q\)) raises the carry cost of long index overlays used by issuers and asset managers to warehouse market exposure while deals are executed \citep{spdj_ipc_page,spdj_bmv_methodology}.
  \item \textit{Pension and wealth portfolios.} Afores and local managers use IPC futures to control beta around rebalances; with \(r-q\approx 3\%\) p.a., maintaining long overlays has a measurable but manageable carry cost, which falls if Banxico eases in line with disinflation \citep{bloomberg_mx_inflation_2025,mnd_inflation_band_2025}.
  \item \textit{Sectoral earnings sensitivity.} Trade-policy shocks (e.g., prospective auto tariffs) and nearshoring dynamics rotate earnings leadership across consumer, industrial, and materials sectors; derivatives are used to implement temporary tilts while stock-specific catalysts arrive \citep{reuters_tariffs_china_autos_2025,reuters_border_jobs_2025}.
\end{enumerate}

\paragraph{Derivatives-specific incidence.}
\begin{itemize}
  \item \textit{Carry and roll.} With modest contango, long futures positions incur negative roll yield absent price appreciation; short overlays benefit. If Fed cuts arrive and Banxico follows gradually, \(r\downarrow\) compresses \(r-q\) and reduces long-carry drag \citep{reuters_ipc_record_2025}.
  \item \textit{Dividend timing.} Near-expiry basis is sensitive to ex-dividend dates of large constituents; it is used to align rolls with the dividend calendar to minimize slippage. The one-year strip embeds an \emph{expected} \(q\)-path that can diverge from realized distributions.
  \item \textit{Liquidity concentration.} Trading and OI cluster in the front contract; far maturities are marks. Hedgers should stage coverage along liquid nodes to reduce execution and basis risk \citep{mexder_ipc_fut}.
\end{itemize}

\paragraph{Sectoral and policy linkages.}
\begin{itemize}
  \item \textit{Industrials/nearshoring.} Improved logistics and regulatory clarity raise earnings durability and support higher multiples; IPC futures hedge interim market swings while capacity is built.
  \item \textit{Energy and quasi-sovereign complex.} Transparent Pemex funding plans lower tail risk and compress the equity risk premium market-wide; slope effects come via the rate channel \citep{reuters_pemex_plan_2025}.
  \item \textit{Trade regime.} Auto-tariff decisions reprice supply chains and consumer prices, altering margins; managers use index futures plus sector tilts to bridge event risk \citep{reuters_tariffs_china_autos_2025}.
\end{itemize}

\paragraph{Actionable playbook (Mexico focus).}
\begin{enumerate}
  \item \textit{Issuers and treasuries.} Use spot strength to pre-hedge equity-linked issuance with short IPC overlays; collapse overlays post-allocation to lock funding at improved valuations.
  \item \textit{Asset owners (Afores).} Separate beta and carry: maintain target equity beta with front IPC futures and manage \(r-q\) via calendar spreads if a slope compression is anticipated from Fed/Banxico sequencing.
  \item \textit{FX integration.} Pair IPC overlays with USD/MXN hedges when liability currency or benchmark is MXN, recognizing that pro-risk rallies can strengthen MXN and alter MXN-denominated tracking.
  \item \textit{Policy levers.} Reinforce fiscal signaling and Pemex transparency to lower macro risk premia; deepen far-month futures liquidity to reduce basis noise and improve hedge effectiveness across horizons.
\end{enumerate}


\begin{description}
  \item[\textit{Faster Fed easing, gradual Banxico.}] Spot supported; \(r\downarrow\) compresses \(r-q\), flattening contango. Bias: favor long beta with lighter carry drag; calendar flatteners (long near/short far) gain.
  \item[\textit{Tariff escalation or growth scare.}] Spot vulnerability; MXN may weaken. Bias: protective short overlays; pair with FX hedges. \(q\) risk rises if dividend guidance is cut; \(r-q\) may widen if local rates stay firm.
  \item[\textit{Pemex/fiscal credibility gains.}] Risk premia compress; issuance windows improve; slope impact via lower \(r\). Bias: opportunistic issuance and overlay reduction; extend hedge tenors as basis stability improves.
\end{description}


Track (i) implied \(r-q(T)=\frac{1}{T}\ln(F_T/S_0)\) across tenors, (ii) front-month basis vs.\ ex-dividend calendar, (iii) slope changes around Banxico/Fed events, and (iv) liquidity/roll metrics in the front two contracts. These diagnostics are used to translate news into implementable positioning while maintaining MXN-aware risk control.


\subsection{Futures term structure (12 Sep 2025)}

See Figure~\ref{fig:ipc_settlements} in Appendix~A for the IPC futures strip on 12 Sep 2025.


The strip is in \emph{modest contango}. From \textit{SEP-25} to \textit{SEP-26} settlements rise by \(\sim 2{,}009\) points. The one-year slope is
\[
\frac{F_{1\mathrm{Y}}}{F_{0}}-1 \approx \frac{63{,}767}{61{,}758}-1 \approx {3.25\%},
\qquad
\Rightarrow\quad r-q \approx \ln\!\Big(\frac{F_{1\mathrm{Y}}}{S_0}\Big)\approx {3.20\%\ \text{p.a.}}
\]
under risk-neutral pricing. Interpretation: funding exceeds expected dividends by \(\sim 3.2\%\) over the next year.
The curve shifted up almost in parallel on the session (\(+0.23\%\) front; \(+0.30\%\)-\(0.32\%\) deferred), a spot-led move consistent with the news backdrop, with limited change in \((r-q)\).
Trading concentrates in \textit{SEP-25}; far maturities showed near-zero volume and OI. Deferred marks should be treated as carry proxies, not high-conviction levels.

\subsection{Interpretation of one day (12 Sep 2025)}
A small \emph{positive basis} is observed: \(F=61{,}840>F^{*}=61{,}819.36\), a premium of \(20.64\) points (\(\sim {0.07\%}\)). In the fair-value map \(F_T=S_0 e^{(r-q)T}\), this is read as a market-implied carry \(\hat c=\frac{1}{T}\ln(F/S_0)\) that is marginally above the model's \((r-q)\), i.e., a slightly lower effective dividend drag or marginally higher funding over the remaining days. Economically, it signals a \emph{modest pro-risk tilt} amid supportive global conditions, not an arbitrage.


\begin{figure}[h]
  \centering
  \includegraphics[width=0.5\textwidth]{figures/ipc.png}
 \caption{One-day pricing decomposition for IPCU25 on 12 Sep 2025.}
\end{figure}

\begin{figure}[h]
  \centering
  \includegraphics[width=1\textwidth]{figures/ipc_pricing_over_the_week.pdf}
  \caption1{Pricing decomposition over the week for IPCU25 (05-12 Sep 2025).}
\end{figure}

\begin{figure}[h]
  \centering
  \includegraphics[width=0.7\textwidth]{figures/ipc_difference.pdf}
  \caption{Difference between market and no-carry fair over the week for IPCU25.}
\end{figure}

\subsection{Weekly term-structure diagnostics (5-12 Sep 2025)}

Across the week, the futures premium to spot \((F/S_0-1)\) drifts from \({+0.40\%}\) (9/5) to \({-0.07\%}\) (9/12), with interim readings of \(+0.27\%\), \(+0.13\%\), \(+0.32\%\), and \(+0.10\%\). The market-model gap (\(F-F^{*}\)) narrows mechanically as \(T\) shortens, while daily \((F,S_0)\) co-move with equities' risk-on tone.


Using the horizon-specific reading \(r-q=\frac{1}{T}\ln(F/S_0)\), the \emph{implied} \((r-q)\) oscillates around small positive values early week and turns slightly negative into 9/12 as \(F\) slips below \(S_0\) on very short \(T\). This is consistent with (i) rising proximity to dividend accruals and (ii) session-specific microstructure (basis and last/settle timing). Given the short horizons, annualized inferences are volatile; the one-year strip discussed above is the appropriate gauge for the structural \(r-q\) of \(\sim 3.2\%\).


The pattern matches the backdrop: Fed-cut expectations lift spot and compress forward premia; Banxico's gradualism maintains positive carry at longer maturities; dividend timing and index-lending dynamics dominate the front-days basis. Net effect: a curve that is \emph{up in level} with \emph{modest contango} intact, while near-expiry prints toggle around flat as ex-dividend proximity and microstructure bite.

\subsection{Does this benefit Mexico? What should be done? How does it affect it?}
It benefits Mexico that the IPC is at record highs with a modest contango. Higher spot IPC levels support wealth effects and lower equity risk premia, aiding funding conditions for issuers. A modest contango implies limited carry costs for maintaining long futures exposure and stable hedging for pension funds and domestic managers.

\paragraph{Investor actions (policy-neutral).}
\begin{enumerate}
  \item \emph{Decompose risks.} Manage spot beta and carry separately. Use front contracts for beta; use calendar spreads to express views on \((r-q)\) compression if Fed cuts and dividend seasonality are expected to flatten the slope.
  \item \emph{Hedge design.} For MXN-based portfolios, pair IPC futures with USD/MXN overlays when relevant to stabilize peso-denominated performance, recognizing the FX-equity interaction in risk-off episodes.
  \item \emph{Dividend timing.} Near-expiry basis is sensitive to ex-div schedules. Align rebalancing and roll dates with the dividend calendar to minimize slippage.
\end{enumerate}

\paragraph{Policy and operating implications.}
\begin{enumerate}
  \item \emph{Signal clarity.} Transparent fiscal guidance and Pemex execution reduce sovereign risk premia, supporting spot and compressing \((r-q)\) via the rate channel.
  \item \emph{Market depth.} Measures that deepen far-month liquidity reduce basis noise and improve hedge effectiveness across horizons.
  \item \emph{Macro mix.} A gradual Banxico path consistent with nominal anchors maintains credible carry while avoiding abrupt slope shifts that raise hedging costs.
\end{enumerate}

\medskip
Into mid-September 2025, the IPC exhibits a spot-led advance with a modest, orderly contango. One-day and weekly diagnostics are coherent with the news flow: supportive global rates and steady local policy compress front premia while leaving the one-year \(r-q\) near \(\sim 3.2\%\). For Mexico, this configuration improves funding optics and portfolio transmission; for investors, it favors disciplined beta plus carry-aware implementation; for policymakers, credibility and market depth are the levers that translate favorable conditions into durable risk premia compression.




\section{TIIE}

\subsection{Recent developments}
Policy signaling and inflation data point to a gradual easing cycle in Mexico, conditioned by external monetary policy and exchange-rate dynamics. On \emph{Aug 7, 2025}, Banco de México lowered the policy rate to \textit{7.75\%} in a divided vote, noting headline inflation around the target band and weak activity \citep{reuters_banxico_cut_aug25,banxico_mps_aug7_2025}. Into mid-September, softer U.S.\ labor data raised the probability of a Federal Reserve cut at the September meeting, weakening the USD and easing global financial conditions \citep{reuters_global_cuts_sep11a,reuters_global_cuts_sep11b}. MXN strengthened and Mexican equities set record highs, consistent with easier external conditions \citep{reuters_mxn_sept15}. These forces are typically transmitted to the local front end: \emph{Funding TIIE} (F-TIIE) fixes in the near term, and the \emph{term structure} of implied monthly compounded F-TIIE via futures \citep{frbny_sofr_page,fred_sofr}.

\subsection{Spot \& futures}
The \emph{Overnight Funding TIIE} is a transaction-based reference rate calculated from peso repo trades settled at INDEVAL; Banxico also publishes compounding indices on business and calendar bases that allow exact accrual over arbitrary periods \citep{banxico_f_tiie_method,banxico_indices_page,banxico_on_method_en}. CME's \emph{Mexican Funding TIIE (Monthly)} futures settle to the annualized compounded average of each day's F-TIIE over the contract month, quoted as
\[
\textit{Index} \equiv 100 - 100\,R_{\text{comp}},
\]
with ACT/360 convention and final settlement equal to \(100 - R_{\text{month}}\) \citep{cme_tiie_monthly_overview,cme_tiie_monthly_specs,cme_tiie_monthly_method,cme_tiie_quarterly_overview}. Hence, a higher futures price implies a lower \emph{implied} compounded monthly rate, and a 1~bp change in the implied rate corresponds to \(\pm \textit{MXN 200}\) per contract (tick \(=0.5\)~bp \(=\) MXN~100) \citep{cme_tiie_monthly_specs}.

\subsection{Mexico-linked implications}
Because F-TIIE underpins Bondes~F and cascades into bank funding curves, the monthly compounded futures are used to transfer near-term policy-path risk from balance sheets to the market \citep{cme_tiie_monthly_overview,cme_ftiie_article}. Easing expectations raise the index (lower implied \(R\)), delivering mark-to-market gains to longs; conversely, stickier inflation or MXN weakness can reprice the front months lower. The curve's \emph{shape} encodes the distribution of expected Banxico cuts across calendar months; microstructure matters because some listed months are \emph{marks to curve} rather than traded prints.


Monthly F--TIIE futures translate into prices (i) the expected path of Banco de México's policy stance, (ii) wholesale MXN funding dynamics, and (iii) external shocks (Fed, USD). By convention,
\[
\text{Futures Price} \equiv 100 - 100\,R_{\text{comp}},
\]
so higher price $\Leftrightarrow$ lower implied monthly compounded rate. Contract sensitivity: $\Delta \mathrm{P\&L} \approx 50{,}000 \times \Delta \text{Index}$ MXN per contract; $1$ bp in $R$ $\Rightarrow \pm$ MXN $200$; minimum tick $=0.5$ bp $=$ MXN $100$.

\paragraph{Banks (ALM and net interest margins).}
\begin{itemize}
  \item \textit{Hedging.} Long monthly F--TIIE futures (benefit from price up) protect margins in a cuts repricing; shorts protect against a hawkish reversal. Timing risk is better expressed with \emph{calendar spreads} (e.g., OCT/NOV) than outright notional.
  \item \textit{Basis risk.} Futures settle to the \emph{monthly} compounded overnight index (ACT/360), while effective funding mixes business-day patterns, liquidity buffers, and idiosyncratic spreads. Manage this basis via limits, backtesting, and mapping rules from F--TIIE to transfer-pricing curves.
\end{itemize}

\paragraph{Pension funds, insurers, and mutual funds.}
\begin{itemize}
  \item \textit{Tactical duration and carry.} Falling implied $R$ supports long-carry in Mbonos/Bondes F. Monthly futures provide a clean front-end overlay to \emph{nowcast} policy moves without disturbing strategic allocations.
  \item \textit{Execution.} Concentrate orders in high-OI near months; predefine roll windows and size caps to mitigate market impact.
\end{itemize}

\paragraph{Corporates and treasuries.}
\begin{itemize}
  \item \textit{Financing cost.} Easing reduces MXN floating-rate expense and expands debt capacity. Align hedges (long monthly F--TIIE) with months of peak cash interest outflow.
  \item \textit{Dollarized revenues.} For USD-revenue firms with MXN liabilities, pair long F--TIIE with FX hedges to stabilize MXN margins.
\end{itemize}

\paragraph{Households and consumption.}
\begin{itemize}
  \item Variable-rate credit costs ease gradually (cards, payroll, auto). Mortgage transmission is slower. The debt-service relief supports consumption with lags; monitor inflation risks via MXN.
\end{itemize}

\paragraph{Sovereign and state-owned enterprises.}
\begin{itemize}
  \item \textit{Rollover costs.} Lower front-end rates reduce near-term funding expense and anchor Bondes F auctions. Staggered hedging across liquid months minimizes slippage and basis volatility.
  \item \textit{Governance.} Neutral tax and accounting treatment of hedge outcomes preserves incentives for prudent use of monthly futures.
\end{itemize}

\paragraph{Money-market and microstructure considerations.}
\begin{itemize}
  \item \textit{Path extraction.} The monthly futures curve encodes \emph{when} and \emph{how much} easing is priced. Kinks with low open interest are often \emph{marks}, not firm views.
  \item \textit{Conventions.} Weekend/holiday compounding extends the last fix and can generate small inter-month differences; document ACT/360 assumptions in internal pricing policy.
\end{itemize}

\paragraph{External channel and MXN.}
\begin{itemize}
  \item Fed cuts typically ease global financial conditions and weaken USD. MXN appreciation lowers tradables inflation and permits Banxico to ease with less risk to expectations. Adverse USD shocks invert this logic and can delay the local cycle.
\end{itemize}

\paragraph{Operational scenarios.}
\begin{itemize}
  \item \textit{Faster-than-expected cuts.} Futures prices rise in the front cluster; longs benefit; liability managers gain relief. Tighten limits on F--TIIE vs.\ internal funding basis and re-estimate repricing betas.
  \item \textit{Delayed or data-dependent path.} Prices fall; shorts protect NIM and margins. Use \emph{calendar spreads} to express “delay” without overcommitting gross notional.
  \item \textit{FX shock (MXN weaker).} Harder easing path; hedge both $R$ and FX to bound MXN P\&L.
\end{itemize}



\subsection{Futures term structure (12 Sep 2025)}

See Figure~\ref{fig:tiie_settlements} in Appendix~A for the TIE futures strip on 12 Sep 2025.

\(\text{Implied }R = 100 - \text{Settle}\). The observed strip (CME monthly) shows a clear easing path from the front:
\[
\text{SEP-25 } \sim {7.74\%}\ \to\ \text{OCT-25 } \sim {7.53\%}\ \to\ \text{NOV-25 } \sim {7.40\%}\ \to\ \text{DEC-25 } \sim {7.37\%},
\]
flattening into 2026 near \(7.0\%\), troughing around \(6.83\%\) by mid-late 2027, then a mild re-steepening \citep{cme_tiie_monthly_quotes}. This configuration is consistent with clustered cuts over \(\text{Q4-2025}\) followed by a slower cadence, not a sharp overshoot. Liquidity is concentrated in the nearest listed months; far maturities should be treated as curve marks \citep{cme_tiie_monthly_overview,cme_tiie_monthly_quotes}.

\subsection{Interpretation of \texorpdfstring{TIEU25}{TIEU25} (one-day read, 12 Sep 2025)}

\begin{figure}[h]
  \centering
  \includegraphics[width=0.5\textwidth]{figures/tiie.png}
 \caption{One-day pricing decomposition for TIEU25 on 12 Sep 2025.}
\end{figure}


\[
\text{Settle} = {93.035}\ \Rightarrow\ R_{\text{mkt}}={6.965\%}.
\]
The fair \((F^{*})\) built by extending the latest daily fix across the horizon implies
\[
F^{*}={91.6369}\quad\Rightarrow\quad R_{\text{mod}}={8.363\%}.
\]
Hence, the \emph{price basis} \(F-F^{*}={+1.398}\) index points maps to an \emph{expected-path gap} of \(\sim{140}\)~bp: the market discounts a monthly-compounded average materially below the flat-at-spot benchmark. Economically, this is a \emph{front-loaded easing} signal: with days already accrued near \(8.01\%\), the remaining path must average lower to deliver \(6.97\%\) for the target month \citep{banxico_on_method_en,cme_tiie_monthly_method}.

\subsection{Does this benefit Mexico? What should be done?}

\begin{itemize}

\item \textit{Banks.} A lower expected path compresses deposit and wholesale funding costs first; asset yields on variable-rate loans reprice with lags. Net interest margins narrow unless liabilities reprice faster. Use front-month \textit{short} F-TIIE futures to protect NIM against a hawkish repricing; use \textit{long} positions if liability relief is the objective \citep{cme_ftiie_article}.

\item \textit{Corporates.} For MXN debt, easing lowers interest expense and raises debt capacity. Treasury teams should pair MXN OIS or monthly F-TIIE longs with FX overlays if revenues are USD-linked.

\item \textit{Households/consumers.} Variable-rate credit costs ease gradually; mortgage and durable-goods financing become more affordable. Inflation risks from MXN swings should be monitored.

\item \textit{Sovereign/Pemes.} Lower front-end rates reduce rollover costs. Transparent hedge governance can lock favorable funding while avoiding concentration in a single month.
\end{itemize}

It is recommended to:
(i) Express a \emph{cuts} view by going \textit{long} near-month contracts; 
(ii) express \emph{timing} via calendar spreads (e.g., OCT/NOV) because monthly comp averages dilute within-month decisions; 
(iii) manage execution in months with open interest; 
(iv) track MXN and U.S.\ rates as primary catalysts.

\medskip
The weekly drift higher in price and the sustained positive basis versus a flat-at-spot model are consistent with an easing path concentrated in \(\text{Q4-2025}\) and moderating thereafter. For Mexico, this configuration supports a measured decline in funding costs without destabilizing curve kinks. Implementation should separate \emph{level} risk (the implied rate \(R\)) from \emph{timing} risk (month-by-month comp effects) and align hedge tenors with liquidity nodes.




\section{MXN/USD}

\subsection{Recent developments}
The peso entered mid-September 2025 near its strongest levels in over a year (\(\sim\)18.4-18.6 USD/MXN), supported by improved global risk appetite and rising odds of U.S. rate cuts. Market focus has been on: (i) the Federal Reserve path toward easing, which compresses U.S.-Mexico rate differentials and is typically MXN-supportive; (ii) Banxico's gradualism, which preserves domestic carry; (iii) fiscal guidance that points to a narrower 2026 deficit; (iv) trade-policy proposals on autos that raise tail risks for the external sector; and (v) evolving Pemex support that can compress sovereign-linked risk premia if execution holds \citep{reuters_usdmxn_quote,reuters_mx_markets_11sep,reuters_mx_markets_12sep,reuters_fed_poll_2025,reuters_cenbank_graphic_2025,reuters_banxico_jun26_2025,banxico_calendar_2025,reuters_budget_2026_2025,reuters_tariffs_autos_2025,reuters_pemex_plan_2025,reuters_fitch_pemex_2025}.

\subsection{Spot \& futures}
Quoting follows CME convention: \emph{price = USD per MXN} (\$/MXN). To obtain the usual USD/MXN, invert. Spot (\(S\)) summarizes contemporaneous USD conditions and Mexico's macro risk. Futures/forwards transfer FX risk across time under \emph{covered interest parity} (CIP):
\[
F = S\,e^{(r_{\mathrm{USD}}-r_{\mathrm{MXN}})T},
\]
with \(r_{\mathrm{USD}}\) a USD risk-free proxy (SOFR) and \(r_{\mathrm{MXN}}\) an MXN money-market rate. Because \(r_{\mathrm{MXN}}>r_{\mathrm{USD}}\) in typical regimes, \(\$/{\rm MXN}\) forwards price \emph{below} spot (equivalently, USD/MXN forwards \emph{above} spot). Deviations at very short tenors can reflect fixing times, holidays, and margin conventions; persistent deviations signal cross-currency basis \citep{bis_cip_2016,bis_cip_2024,frbny_sofr,frbny_sofr_index,cme_fx_overview,cme_mxn_product,cme_mxn_rulebook}.

\subsection{Mexico-linked implications}
FX levels and forward structure transmit into inflation, real incomes, and public finances. A firm MXN lowers traded-goods inflation and import costs, but compresses peso revenues for USD-earning sectors unless hedged. For the sovereign and large corporates, credible fiscal signals and Pemex risk reduction compress term premia that feed into longer-dated forwards. For real-money and corporate treasuries, CIP-implied discounts in \(\$/{\rm MXN}\) encode carry; hedge design should pair FX with underlying commodity or rate exposures to stabilize peso cash flows.

\subsection{Futures term structure (12 Sep 2025)}

See Figure~\ref{fig:mxnusd_settlements} in Appendix~A for the MP futures strip on 12 Sep 2025.

The curve is \emph{downward} in \(\$/{\rm MXN}\) (\(F<S\)), which corresponds to an \emph{upward} USD/MXN forward path. Annualized, SEP-25\(\to\)MAR-27 implies \(\sim 4.0\%\) p.a. MXN depreciation under CIP, consistent with \(r_{\mathrm{MXN}}-r_{\mathrm{USD}}\) for those tenors.

Activity concentrates in SEP- and DEC-25 (high volume and OI). Thinly traded months are marks-to-curve; treat small kinks with caution. With the curve sloping down in \(\$/{\rm MXN}\), long-MXN positions (long futures) face \emph{negative roll}, while short-MXN (long USD) earns forward carry.

\subsection{Interpretation of \texorpdfstring{MPZ25}{MPZ25} (one-day read)}

\begin{figure}[h]
  \centering
  \includegraphics[width=0.5\textwidth]{figures/usdmxn.png}
\caption{One-day pricing decomposition for MPZ25 on 12 Sep 2025.}
\end{figure}

Using \(\$/{\rm MXN}\) spot \(S=0.05427\), front settlement \(F=0.05421\), and \(T=3/360\), the forward sits slightly \emph{below} spot by \(-6\times 10^{-5}\) \$/MXN (\(-0.11\%\) of spot). Under CIP:
\[
\frac{F}{S}=e^{(r_{\mathrm{USD}}-r_{\mathrm{MXN}})T}\;\Rightarrow\;
r_{\mathrm{MXN}}-r_{\mathrm{USD}}=\frac{1}{T}\ln\!\Big(\frac{S}{F}\Big).
\]
At such short horizons this annualizes to a large number due to \(T\ll 1\); economically it simply states \(r_{\mathrm{MXN}}>r_{\mathrm{USD}}\). The magnitude is well within no-arbitrage bands once day-counts, holiday effects, and bid-ask are recognized. Directionally, the sign matches the macro backdrop: Fed-easing expectations and Banxico gradualism favor a small forward \emph{discount} in \(\$/{\rm MXN}\).

\subsection{Does this benefit Mexico? What should be done?}

A firm MXN with CIP-consistent forward discounts lowers imported-inflation pressure and stabilizes local financing conditions. For exporters and remitters, the stronger currency tightens peso revenues, which increases the importance of hedge overlays. The observed structure is benign: it rewards USD-based investors who hedge into MXN (positive carry) and allows domestic treasuries to pre-fund USD at predictable forward points.

\paragraph{Actions (policy-neutral investors).}
\begin{enumerate}
  \item \textit{Separate level from carry.} Manage spot risk (USD/MXN) and carry risk (forward points) independently. Use short-dated forwards for cash-flow alignment; employ calendar rolls to manage carry.
  \item \textit{Pair hedges.} Combine FX hedges with commodity or rate hedges where revenues or costs are joint (e.g., silver exporters hedging both XAG and MXN) to stabilize MXN cash flows.
  \item \textit{Tenor discipline.} Favor liquid fronts (SEP, DEC) for execution; stair-step coverage to reduce fixing risk. Avoid annualizing micro-tenor signals.
\end{enumerate}

\paragraph{Actions (Mexico's operating environment).}
\begin{enumerate}
  \item \textit{Credible fiscal and Pemex execution.} Sustained consolidation and transparent Pemex support compress term premia embedded in longer-dated forwards.
  \item \textit{Hedge accounting clarity.} Neutral tax and accounting treatment for FX hedges reduces frictions and encourages systematic risk transfer.
  \item \textit{Market plumbing.} Robust local money markets and settlement infrastructure shrink cross-currency basis and improve CIP pass-through.
\end{enumerate}



\section{Bibliography}
\bibliography{refs}


\section{Appendix A: \\ Price Settlements \\ (Sep 12, 2025)}
\newpage

%% add another pdf
\begin{figure}[h]
  \centering
  \includegraphics[width=0.99\textwidth]{appendix/CORN12SEP.pdf}
  \caption{Futures Settlements for Corn (12 Sep 2025)}
  \label{fig:corn_settlements}
\end{figure}

\begin{figure}[h]
  \centering
  \includegraphics[width=0.99\textwidth]{appendix/CRUDEOIL12SEP.pdf}
  \caption{Futures Settlements for Crude Oil (12 Sep 2025)}
  \label{fig:crudeoil_settlements}
\end{figure}

\begin{figure}[h]
  \centering
  \includegraphics[width=0.99\textwidth]{appendix/SILVER12SEP.pdf}
  \caption{Futures Settlements for Silver (12 Sep 2025)}
  \label{fig:silver_settlements}
\end{figure}


\begin{figure}[h]
  \centering
  \includegraphics[width=0.99\textwidth]{appendix/IPC12SEP.pdf}
  \caption{Futures Settlements for IPC (12 Sep 2025)}
  \label{fig:ipc_settlements}
\end{figure}


\begin{figure}[h]
  \centering
  \includegraphics[width=0.99\textwidth]{appendix/TIIE12SEP.pdf}
  \caption{Futures Settlements for TIIE (12 Sep 2025)}
  \label{fig:tiie_settlements}
\end{figure}



\begin{figure}[h]
  \centering
  \includegraphics[width=0.99\textwidth]{appendix/MXNUSD12SEP.pdf}
  \caption{Futures Settlements for MXN/USD (12 Sep 2025)}
  \label{fig:mxnusd_settlements}
\end{figure}

\section{Appendix B: \\ Python Notebook }
\includepdf[pages=2-]{../nappendix/appendixb.pdf}

\end{document}